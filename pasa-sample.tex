\documentclass{pasa}%

\usepackage{graphicx,color}
\newcommand{\re}[1]{\textcolor{blue}{[{\bf RE}: #1]}}


\title[The Core-Cusp Problem Revisited: ULDM vs. CDM]{The Core-Cusp Problem Revisited: ULDM vs. CDM}

%% Please note that the command \and is not supported in \author.
\author[Emily Kendall and Richard Easther]{Emily Kendall$^1$, Richard Easther$^1$, \affil{$^1$Department of Physics, University of Auckland, Private Bag 92019, Auckland, New Zealand}}%


\jid{PASA}
\doi{10.1017/pas.\the\year.xxx}
\jyear{\the\year}


\usepackage{aas_macros}
\usepackage{hyperref} 
\hypersetup{colorlinks,citecolor=blue,linkcolor=blue,urlcolor=blue}

%%%%%%% IMPORTANT: We disable hyperlinks by default with this line, to avoid the error "\pdfendlink ended up in different nesting level" while writing.
\hypersetup{draft}
%%%%%%% You may comment or delete the line above to make hyperlinks in your paper active. If you then encounter a strange "\pdfendlink ended up in different nesting level than \pdfstartlink", don't worry! Uncomment the line again, and see https://www.overleaf.com/help/246 for further information.
% \usepackage{natbib}
\bibliographystyle{abbrvnat}
\setcitestyle{numbers,open={[},close={]}}

\begin{document}

\begin{frontmatter}
\maketitle

\begin{abstract}
The core-cusp problem is often cited as a motivation for the exploration of dark matter models beyond standard CDM [cold dark matter]. One such alternative is ULDM [ultra-light dark matter]; extremely light scalar particles exhibiting wavelike properties on kiloparsec scales. ULDM dynamics are governed by the Schr\"{o}dinger-Poisson equations whose solitonic ground state solutions consist of gravitationally-bound condensates. Astrophysically realistic ULDM halos consist of larger NFW-like configurations with a possible solitonic core and associated core-halo mass relation. It is proposed that UDLM may resolve the core-cusp discrepancy without recourse to baryonic physics and we describe a parameterisation for the radial density profiles of ULDM halos that allows for environmental variability of the core-halo mass relation. We then compare the semi-analytic profiles of ULDM and CDM and conclude that for halos up to $10^{12} M_\odot$ there are feasible ULDM profiles which provide a diminished halo density at astrophysically accessible inner radii, thus ameliorating the core-cusp problem in principle and comparing theoretical profiles to observational data from the SPARC database suggests that this profile can give reasonable fits to various subsets of the data. However, we also  argue that more comprehensive observational data and better simulations with baryonic feedback will be needed before robust conclusions can be drawn regarding the suitability of either model.  \re{what is ``either''; we are really just talking about ULDM? Or it is the relative goodness of fit?}

\end{abstract}

\begin{keywords}
Cosmology -- Core-Cusp Problem 
\end{keywords}
\end{frontmatter}


\section{INTRODUCTION }
\label{sec:intro}

It is widely agreed that non-baryonic dark matter constitutes the majority of the mass of the observable universe, but its precise nature remains an open question. Many dark matter models have been proposed, with particle CDM [Cold Dark Matter]  being the most widely studied. This scenario successfully accounts for the large scale structure of the universe \cite{Springel:2005nw} and the spectrum of anisotropies in the microwave background \cite{deBernardis:2000sbo, Hanany:2000qf, Halverson:2001yy, Netterfield:2001yq, Lee:2001yp, Ade:2015xua,  Hu:2001bc}, but the so-called ``small-scale crisis''  remains a challenge \cite{Weinberg:2013aya}. A key issue is the apparent tension between the  central density profiles of dark matter halos in simulations containing only gravitationally interacting CDM, and those inferred from observational data. Simulations tend to produce `cuspy' central density profiles \cite{Navarro:1995iw}, which grow as $1/r$ at small radii, but observational data appears to favour flattened central cores \cite{Moore:1994yx}. This so-called core-cusp problem has been the focus of much recent attention \cite{Dutton:2018nop, Read:2018pft, Genina:2018}. 
 
The seriousness of the core-cusp problem is the subject of ongoing debate, and may be ameliorated by adding baryonic matter to CDM simulations \cite{Benitez-Llambay:2018}. Nevertheless, the wider category of  ``small-scale'' problems in standard CDM along with tighter constraints from direct-detection experiments \cite{Schumann:2019eaa} motivates the study of alternative dark matter models. One scenario which has gained substantial traction is ultra-light dark matter [ULDM], also known variously as scalar-field dark matter, $\Psi$ dark matter, BEC dark matter and fuzzy dark matter, in which an extremely light scalar field constitutes the principal component of dark matter.  As reviewed by Hui {\em et al.\/} \cite{Hui:2016ltb}, the very small ULDM particle mass ($\mathcal{O}(\sim 10^{-22}eV)$) corresponds to a kiloparsec-scale de Broglie wavelength. ULDM thus exhibits novel wave-like behaviour on astrophysically interesting scales, including soliton-like gravitationally confined Bose-Einstein condensates. ULDM simulations suggest that realistic astrophysical halos have an inner core consisting of a kiloparsec scale condensate, while the outer halo is a virialised system of scalar particles that resembles standard CDM \cite{Schwabe:2016rze, Veltmaat:2018dfz}. 

The NFW [Navarro-Frenk-White] profile \cite{Navarro:1995iw} is characteristic of collisionless CDM, which is most commonly imagined to consist of WIMP [Weakly Interacting Dark Matter] particles \cite{Robles:2018fur}. \re{is this the right citation? May not even need one} The theoretical ULDM profile combines a solitonic inner core, surrounded by an NFW outer halo. The centres of ULDM solitons are flat, so can conceivably resolve the core-cusp problem without reference to baryonic astrophysics.   However, solitonic density profiles obey an inverse mass-radius scaling law, so the density of the ULDM halo might exceed that of an analogous NFW halo for a finite range of small radii for larger galaxies. In particular, Ref~\cite{Robles:2018fur} concludes that NFW profiles can actually outperform ULDM profiles for galaxies with halo masses $M_h \gtrsim 10^{11} M_{\odot}$, or ``large dwarfs'', even in the absence of baryons. \re{is there an upper limit on a large dwarf?}  




We examine the effect of scatter in the core-halo mass scaling relation and its implications for discussions of the core-cusp discrepancy in ULDM. Starting from the semi-analytic density profile of Ref.~\cite{Robles:2018fur} we look at the scatter in the parameters  implied by Ref.~\cite{Schive:2014hza}, showing that it may ease concerns that the core-cusp problem is  exacerbated for ULDM relative to CDM for large dwarf galaxies. Further, the incoherent outer regions of ULDM halos are subject to strong fluctuations, both temporally and spatially. These are not captured by semi-analytic halo density profiles and we argue that these fluctuations  may accentuate the intrinsic scatter in halo parameters. Finally,  baryonic feedback is known to be significant for dwarf galaxies \cite{2018MNRAS.473.5698D, Benitez-Llambay:2018} and neither the NFW or ULDM profiles incorporates its effects. 

With these limitations in mind, we caution against attempting to discriminate between the applicability of ULDM and CDM models based on DM-only simplified theoretical profiles. Indeed, we find that for DM-only models, neither ULDM halos (for ULDM particle mass $0.8-2.5\times 10^{-22} \operatorname{eV}$) nor NFW halos provide an overwhelmingly convincing fit to rotation curves of large dwarf galaxies, suggesting that significant contributions from baryonic physics are needed for either model to yield a good fit to rotation curves from the SPARC database \cite{Lelli:2016zqa}. Moreover,  many curves are extracted from a few data points over small range of radial distances and many profiles have significant uncertainties, further complicating attempts to draw robust conclusions.  

We show that  the ULDM model seemingly ameliorates the core-cusp problem in galaxies  exhibiting a steep decrease in rotation velocity at small radii if the ULDM mass is of order $10^{-23}\operatorname{eV}$. However, a mass this small is in tension with other   constraints and a multifaceted investigation is needed to determine whether ULDM successfully describes galactic density profiles. Consequently, we argue that the primary conclusion to be drawn from this parameter-fitting exercise is that the analyses of the core-cusp problem -- and potentially other ``small scale'' anomalies -- based on simplified DM-only distributions are not likely to meaningfully test these scenarios, especially when  observational data is limited and detailed numerical simulations have not been performed. 

The structure of the paper is as follows. In Section \ref{sec:models}, we review the construction of semi-analytic density profiles for both the ULDM and CDM models. and briefly discuss aspects of realistic ULDM halos which are not captured by the semi-analytic model. In Section \ref{sec:velocity} we compare the semi-analytic density profiles for ULDM and CDM halos in the dwarf galaxy mass range $10^{11} - 10^{12}\operatorname{M}_{\odot}$, taking into account statistical variation in both the NFW concentration parameter and the ULDM core-halo mass relation. We then compare the radial velocity profiles inferred from these density profiles with astrophysical data from the SPARC database \cite{Lelli:2016zqa}. We conclude in Section \ref{sec:conclusion}.

 
\section{Semi-analytic models of dark matter halos}\label{sec:models}


\subsection{The NFW profile of CDM}\label{sec:NFW}

We begin by looking at the semi-analytic parametrisations of ULDM and CDM halo models. The  well known  NFW   profile of CDM \cite{Navarro:1995iw, Maccio:2008pcd}  is given by
%
\begin{equation}\label{eq:nfw}
    \rho_\mathrm{NFW}(r)=\frac{\rho_0}{\frac{r}{R_s}\left(1+\frac{r}{R_s}\right)^2} \, .
\end{equation}
%
At small radii the profile is proportional to $1/r$, while at large radii it goes as $1/r^3$. The parameters $\rho_0$ and $R_s$ vary from halo to halo; $\rho_0$ can be interpreted as a characteristic density, while $R_s$ is the scale radius and determines the distance from the centre at which the transition between the `small $r$' and `large $r$' limits occurs. 

The NFW halo is assumed to be radially symmetric, and requires truncation at a finite radius in order to prevent the mass integral diverging as $r\rightarrow \infty$. This is typically set by the virial radius, which is  determined approximately via the spherical top-hat collapse model \cite{White:2000jv, Suto:2015jdt, Herrera:2017epn}, describing the evolution of a uniform spherical overdensity in a smooth expanding background. Gravitational collapse of the overdensity halts when virial equilibrium is reached. The corresponding virial radius is the radius at which the mean internal density is $\Delta_c \rho_\mathrm{crit}(t)$. Here $\rho_\mathrm{crit}(t)$ is the critical density of the universe at time $t$. The  factor $\Delta_c$ is of order $10^2$ and while different conventions exist, we will make the common choice of $\Delta_c = 200$ \cite{Richings:2018} in what follows. 

Once the virial radius is specified as the outer limit of the halo, equation \ref{eq:nfw} completely determines the density profile for given  $\rho_0$ and $R_s$. For any given virial mass, there is a range of corresponding NFW density profiles, with the distributions of $\rho_0$ and $R_s$ emerging from the mass-concentration-redshift relation seen in N-body simulations and observations \cite{Ludlow:2013vxa, Ragagnin:2018enf}. 

\subsection{The piecewise ULDM halo profile}

ULDM dynamics is governed by the Schr{\"o}dinger-Poisson system of coupled differential equations. In a static background, they take the dimensionless form  
%
\begin{align}
    &i\dot{\psi} = -\frac{1}{2}\nabla^2\psi+\Phi\psi \\
    &\nabla^2\Phi = 4\pi \vert \psi\vert^2
\end{align}
%
where $\psi$ is the ULDM wavefunction, $\Phi$ is the Newtonian potential, and the density $\rho \propto |\psi|^2$. The solitonic ground state profile cannot be written down analytically, but given a numerically computed spherically symmetric  profile $\psi$ with $\psi(0)=1$, the full family of solutions is
%
\begin{equation}
    \psi'(x) = \gamma\psi(\sqrt{\gamma}x),
\end{equation}
%
where $\gamma$ is a scaling parameter and the dimensionless mass of the soliton is proportional to $\sqrt{\gamma}$, while the dimensionless radius is proportional to $1/\sqrt{\gamma}$. The dimensionless density $\vert\psi\vert^2$ and dimensionless radius $x$ can be transformed into dimensionful quantities by
\begin{align}
    \rho &= \mathcal{M}\mathcal{L}^{-3}\vert\psi\vert^2, \label{eq:density_conv} \\
    r &= \mathcal{L}x, \label{eq:mass_conv}
\end{align}
where
\begin{equation}\label{eq:length}
    \mathcal{L}=\left(\frac{8\pi\hbar^2}{3 m^2H_0^2\Omega_{m_0}}\right)^{\frac{1}{4}}\approx121\left(\frac{10^{-23}\operatorname{eV}}{m}\right)^{\frac{1}{2}}\operatorname{kpc},
\end{equation}
%
and 
%
\begin{align}\label{eq:mass}
    \mathcal{M}&=\frac{1}{G}\left(\frac{8\pi}{3 H_0^2\Omega_{m_0}}\right)^{-\frac{1}{4}}\left(\frac{\hbar}{m}\right)^{\frac{3}{2}}\nonumber\\
    &\approx 7\times 10^7\left(\frac{10^{-23}\operatorname{eV}}{m}\right)^{\frac{3}{2}}\operatorname{M}_{\odot}.
\end{align}



 
Ref.~\cite{Robles:2018fur} gives a piecewise parameterization of the generic ULDM profile 
%
\begin{equation}\label{eq:piecewise}
     \rho(r)=
    \begin{cases}
      \rho_{sol}(r), & 0\leq r \leq r_{\alpha} \\
      \rho_\mathrm{NFW}(r), & r_{\alpha}\leq r \leq r_{\mathrm{vir}},
    \end{cases}
\end{equation}
%
where $\rho_{sol}(r)$ is the appropriately scaled density profile of the ground state soliton solution. The contribution from the solitionic core and the overall virial mass is predicted to obey a scaling relationship \cite{Schive:2014hza, Chavanis:2019faf}, which sets the central density, $\rho_c$, of a ULDM halo with virial mass, $M_{\mathrm{vir}}$. This yields an expression relating the core size to the velocity dispersion, and finally to the halo virial mass.%
\footnote{The authors of \cite{Schive:2014hza} suggest the following general expression:
\begin{equation}
    M_c = \alpha \left(\vert E\vert/M\right)^{1/2},
\end{equation}
where the core mass $M_c$ is determined by the total energy, $E$, and the total mass of the halo, $M$ where $\alpha$ is a constant of order unity. They then explain that the right hand side of the equation represents the halo velocity dispersion, while the left hand side  represents the inverse core size due to soliton scaling laws. By invoking the virial condition of the spherical collapse model, the authors then  construct the redshift dependent relationship between the solitonic core mass and the halo virial mass for a ULDM halo.}

\begin{figure}[t]
\centering
\includegraphics[scale = 0.6, trim={1cm 0cm 0cm 0.4cm}]{slice.eps}
\caption{Illustration of the scale of the fluctuations present in the incoherent outer halo for a merger of 8 randomly located solitons. The contour plot represents the ($\mathrm{log}_{10}$ scaled) local density across a slice through the centre of the final halo. In this plot, distance is not log-scaled, and we see that the spatial size of the fluctuations is of the same order of magnitude as the solitonic core itself.}\label{fig:contour}
\end{figure}
%
The core-halo mass relation can also be understood simply by matching the virial velocities of the core and the wider halo (see Appendix \ref{app:core-halo} for details). 
At $z=0$ the relationship is found to be \cite{Schive:2014hza} 
%
\begin{equation}\label{eq:central_dens}
    \rho_c = 2.94\times10^6 \operatorname{M}_{\odot}\operatorname{kpc}^{-3}\left(\frac{M_{\mathrm{vir}}}{10^9 M_{\odot}}\right)^{4/3}m_{22}^{2},
\end{equation}
and 
\begin{equation}
    r_c = 1.6 \operatorname{kpc}\left(\frac{M_{\mathrm{vir}}}{10^9 M_{\odot}}\right)^{-1/3}\frac{1}{m_{22}},
\end{equation}
%
where $r_c$ is the radius at which the density is half of the central value, and $m_{22}$ is given by $m_{22} \equiv m / 10^{-22} \operatorname{eV}$ where $m$ is the ULDM particle mass. 

While the piecewise semi-analytic ULDM profile is a useful tool, one should be mindful of its inherent limitations. For example, while a number of studies have attempted to establish the `universal' properties of ULDM halos, many of these studies generated ULDM halos through the merger of several smaller compact objects such as solitons \cite{Schwabe:2016rze, Mocz:2017wlg}. This method of assembling halos is not necessarily representative of a realistic structure formation process. However, these methods of simulating ULDM halos persist due to the computational difficulty of undertaking large-scale cosmological simulations. For this reason we have limited information from which to draw robust conclusions about the properties of ULDM halos originating due to gravitational collapse in the early universe. In particular, more work is needed to understand the characteristic timescales associated with the formation of quantum pressure supported cores in different scenarios, including condensation from a fluctuating background, gravitational collapse in an expanding background, and mergers of objects with and without stable central cores. Until the dynamical timescales of core formation are fully understood for  a wide range of halo masses, we must be cautious when extrapolating the core-halo mass relation assumed by this semi-analytic halo model to physical halos in regions of parameter space which have yet to be explored numerically. Moreover, it is difficult to accurately predict the effect that baryonic feedback will have on the formation of solitonic cores in halos of different masses, which could be significant at small radii in the  present context.

Halo substructure is likewise missing from the semi-analytic model presented above. In simulations of soliton mergers the resulting halos have turbulent outer regions, with fluctuations on scales comparable to the core size, as illustrated in Figure \ref{fig:contour}. In addition to the fluctuations inherent in a large ULDM halo, smaller halos are likely to orbit or interact with larger halos. This sort  substructure is not captured by the semi-analytic model described here, and their predictions for tracer velocity profiles may thus not match those of realistic astrophysical objects, particularly when global properties are inferred from a small number of tracer objects. Furthermore, temporal fluctuations in the core density are also missing from the semi-analytic model. Realistic halo cores are not exact soliton solutions of the Schr\"{o}dinger-Poisson equation,  they interact non-trivially with the fluctuating NFW-like outer halo, and their central densities can vary with time by as much as a factor of two \cite{Veltmaat:2018dfz}.

Taken together, these limitations suggest that the core-halo mass relation of the semi-analytic model should  be interpreted as a statement about the averaged characteristics of a  distribution. We attempt to estimate the corresponding variance by   considering a range of possible central densities for a given virial mass,  analogously to the scatter in NFW concentration parameters \cite{Maccio:2008pcd}). We estimate the parameter ranges from the results of Ref.~\cite{Schive:2014hza}, which indicates that   a scatter  in the core mass $M_c$ of up to $\pm 50\%$ is possible, for a given virial mass. Unfortunately, the small sample size and limited halo mass range ($ M_{\mathrm{vir}} \approx 10^8-10^{11} \operatorname{M}_{\odot}$) found in ~\cite{Schive:2014hza}  prevents a detailed analysis of the properties of realistic astrophysical halos  but  future simulations (especially those including baryonic feedback) should lead to improved predictions for this distribution. 

We also allow for variation in the radius at which the solitonic profile of the ULDM halo transitions into an NFW profile. This is acknowledged in Ref~\cite{Robles:2018fur} and is captured by the parameter $\alpha$; the transition radius, $r_{\alpha}$, is given by $r_{\alpha} = \alpha r_c$, where $3 \leq \alpha \leq 4$. The chosen value of the radius of transition will of course affect the parameters of the theoretical outer NFW halo, in particular the scale radius, as the core-halo mass relation should be maintained as the transition radius is varied.

When variation in both the central soliton density and  transition radius are accounted for, we can create a range of plausible ULDM halo profiles for a given halo virial mass by using the virial mass to predict $\rho_c$, and taking the range of densities to be $\pm 50\% $ of this central value. Given a specific central density and $\alpha$, the solitonic piece of the ULDM profile is then completely specified, and its mass can be calculated. The remainder of the virial mass must be accounted for by the NFW tail of the profile. By matching the densities of the NFW tail to the inner soliton at the transition radius, the values of the $R_s$ and $\rho_0$ for the ULDM profile NFW tail are obtained.  

%In summary, we stress that while we have made attempts to account for statistical variation in halo parameters, one cannot expect this simplified semi-analytic model of ULDM halo structure to accurately represent real astrophysical objects, and we must therefore be careful when using such models to draw conclusions as to the relative performance of the ULDM model and the CDM model of dark matter. Likewise, it is known that the NFW profile does not accurately represent CDM models when baryonic feedback is included, and this further limits the utility of a direct comparison of our two semi-analytic profiles with observational data. 


\section{ULDM and CDM halos and astrophysical data}\label{sec:velocity}

\begin{figure*}[t]
\centering
\includegraphics[scale=0.4, trim={2cm 0cm 0cm 1cm}]{new_combined_1.png}
\caption{Density profiles as a function of radius (normalised to the virial radius) for ULDM and NFW halos of masses $10^{11}\operatorname{M}_{\odot}$ (top) and $10^{12}\operatorname{M}_{\odot}$ (bottom). The left panel represents the results for $m_{22} = 0.8$, while the right panel corresponds to $m_{22}=2.5$. The transition radius is fixed at $r_{\alpha} = 3.5*r_c$. The blue shaded region represents the ULDM profile with $\operatorname{M}_c = \operatorname{M}_{\mathrm{cp}} \pm 50 \% \operatorname{M}_{\mathrm{cp}}$, while the solid blue line represents the ULDM profile when the theoretical core-halo mass relation is taken to be exact. The red shaded region represents the range of NFW profiles for a halo of the same virial mass with a 2$\sigma$ variation around the median (solid red line).}\label{fig:profiles}
\end{figure*}
 
We now compare the radial profiles of ULDM halos to NFW halos using the semi-analytic profiles described above, focusing on masses in the range $10^{11}$ and $10^{12} \operatorname{M}_{\odot}$ which may show an apparent worsening of the core-cusp problem \cite{Robles:2018fur}. Figure \ref{fig:profiles} shows compares halos; the shaded blue region represents the ULDM halos for which the core-halo mass relation has a scatter $\operatorname{M}_c = \operatorname{M}_{\mathrm{cp}} \pm 50 \% $ range, where $\operatorname{M}_{\mathrm{cp}}$ is the theoretical predicition for the core mass.

 The  Schr{\"o}dinger-Poisson soliton scaling relations show that this mass range corresponds to a range of $ \gamma_p /4 \leq \gamma \leq 9\gamma_p/4$, where $\gamma_p$ is the theoretical prediction of the square root of the dimensionless central density. Consequently, there is a large variation in the central density and thus widely varying predictions for the  ULDM profiles. We fix $\alpha = 3.5$ (in the middle of the predicted range) which does not affect the central density as the core lies well within the solitonic region. Meanwhile, the red shaded regions of Figure \ref{fig:profiles} show the $2\sigma$ variation about the theoretical prediction for the concentration parameter of the corresponding NFW halo \cite{Maccio:2008pcd}. \re{think  we need to talk about any $\alpha$ dependence in these plots.}



Following Ref~\cite{Robles:2018fur}, we plot to a minimum radius of $r/r_{\mathrm{vir}} = 10^{-4}$ and for the same choices of $m_{22}$. For any $\operatorname{M}_{\mathrm{vir}}$, the NFW halo density  will inevitably exceed that of the ULDM halo at very small radii, though the threshold for this transition may be arbitrarily small, and not observationally relevant. However, the apparent worsening of the core-cusp discrepancy does depend on the choice of inner radial cutoff.

From Figure \ref{fig:profiles} we see that for halo masses of $10^{11}\operatorname{M}_{\odot}$ there is a wide range of $M_c$ for which the ULDM profile is `less cuspy' than its NFW counterpart. For a halo mass of $10^{12}\operatorname{M}_{\odot}$ and a ULDM particle mass $m_{22}=0.8$ the range of plausible ULDM profiles likewise includes those which are `less cuspy' than the corresponding NFW profile. At higher particle mass ($m_{22}=2.5$) for $10^{12}\operatorname{M}_{\odot}$ halos, the NFW profiles tend to be less peaked than corresponding ULDM profiles at radial distances in  the range $10^{-4}\leq r/r_{\mathrm{vir}} \leq 1$.  

We compare these theoretical profiles to observations drawn from the SPARC database.  However, observations yield the (line of sight) velocity distributions of tracer stars as a function of galactocentric radius, rather than the dark matter halo density itself. We must therefore transform our theoretical density profiles into velocity profiles to make the required comparisons. The effects of non-circular motion and kinematic irregularities constitute a non-trivial source of random error in observed velocities, which should be kept in mind especially when working with limited data sets. 


\begin{figure*}[t]
\centering
\includegraphics[scale=0.9, trim={0cm 2.5cm 3cm 0cm}]{000_vs_SPARC_10_12.png}
\caption{Velocity distributions for galaxies with maximum velocities in the range $125 \leq v < 175\operatorname{kms}^{-1}$ in the SPARC database. Data at innermost radii is limited for these galaxies, making it difficult to determine the overall characteristics of the profiles. The SPARC data is plotted along with theoretical NFW and ULDM profiles, assuming a virial mass of $10^{12} \mathrm{M}_{\odot}$, $m_{22} = 2.5$, and $\pm 50 \%$ scatter in the ULDM core-halo mass relation and $\pm2\sigma$ scatter in NFW concentration. Galaxies in the legend are ordered from highest maximum velocity (top) to lowest (bottom).  \re{red v blue?}}\label{fig:high_v} 
\end{figure*}

We convert  density profiles to velocity distributions \cite{Sofue:2008wt} via 
%
\begin{equation}
    V(r)^2 = \frac{4\pi G}{r}\int_0^r \rho(r')r'^2 dr',
\end{equation}
where 
\begin{equation}\label{eq:vel_decomp}
    V^2 = V_{\mathrm{disk}}^2 + V_{\mathrm{bulge}}^2 + V_{\mathrm{gas}}^2 + V_{\mathrm{halo}}^2.
\end{equation}
%
The SPARC database contains photometric data for 175 galaxies and rotation curves from $\mathrm{H}_{\mathrm{I}}$/$\mathrm{H}_{\alpha}$ studies. The disk and bulge velocities in the SPARC database are given for $\Upsilon = 1 \operatorname{M}_{\odot}/\operatorname{L}_{\odot}$ at $3.6\operatorname{\mu m}$. However, the greatest source of uncertainty in mass modelling is the assumed stellar mass-to-light ratio, $\Upsilon_\star$ \cite{Lelli:2016zqa}. As in \cite{Robles:2018fur}, we  assume a constant value of $\Upsilon_\star = 0.2 \operatorname{M}_{\odot}/\operatorname{L}_{\odot}$ at $3.6\operatorname{\mu m}$, likewise noting that  this constitutes a non-trivial source of uncertainty. Moreover, there is significant uncertainty in the SPARC data itself, and error bars are omitted in the following graphs for ease of viewing. 

The characteristics of the velocity profiles in the SPARC database vary widely from galaxy to galaxy, making it difficult to group them into clearly defined subsets but we consider two groups of galaxies; those with maximum tracer velocities $75 \leq v < 125\operatorname{kms}^{-1}$, and those for which $125 \leq v < 175\operatorname{kms}^{-1}$. The former group tends to exhibit a strong steepening in the radial velocity profile toward the inner halo, while the  profiles for thje latter  group contains profiles  are comparatively flat\footnote{We exclude data for which the velocities calculated according to Equation \ref{eq:vel_decomp} are inconsistent - this can occur due to the uncertainty in the assumption for $\Upsilon_\star$.}. We assume that higher asymptotic velocities correspond to a larger halo mass, and consider halo masses in the range $10^{11} - 10^{12} \operatorname{M}_{\odot}$, expecting that masses at the top end of the range will give a better match to galaxies with higher asymptotic velocities. 



 In Figure \ref{fig:high_v}, we see that galaxies with asymptotic velocities at the higher end of the range do not have a clearly pronounced steepening of the velocity profile at small radii, although in some cases there is simply no data at small radii. Choosing a halo mass $10^{12} \mathrm{M}_{\odot}$ gives a reasonable fit to the theoretical models overlap at higher radii. By contrast, for galaxies smaller maximum velocities ($75 \leq v < 125\operatorname{kms}^{-1}$) there is more data at smaller radii and we see the steepening rotation curves characteristic of cored density profiles, as seen in Figure \ref{fig:low_v}. In this case, choosing parameters such that the theoretical profiles overlap with the data at small radii is easy (in this case $m_{22} = 0.1$, $\mathrm{M}_{\mathrm{vir}}5\times10^{11} \mathrm{M}_{\odot}$), however it is not clear whether the behaviour of this profile would fit data at larger radii were it available. However,  a ULDM particle mass $m_{22} = 0.1$ is in tension with constraints from the Lyman-$\alpha$ forest, as well as high-redshift UV luminosity function comparisons \cite{Amendola:2005ad, Bozek:2014uqa, Armengaud:2017nkf, Ni:2019qfa, Nebrin:2018vqt}. However, since baryonic feedback has the greatest impact in the innermost regions of realistic halos, agreement between the semi-analytic model and the small radius observational data should be interpreted cautiously, especially since this is also the region where assumptions regarding the stellar mass to light ratio have the greatest significance.  \re{still not sure about this para; edited to compress it to what I  THINK you are saying} \re{Also, why these particular values of $m_{22}$?}

\begin{figure*}[t]
\centering
\includegraphics[scale=0.9, trim={0.5cm 1cm 0cm 1cm}]{000_vs_SPARC_5_10_11.png}
\caption{Velocity distributions for galaxies with maximum velocities in the range $75 \leq v < 125\operatorname{kms}^{-1}$ in the SPARC database. Data at outer radii is limited for these galaxies, making it difficult to determine the overall characteristics of the profiles. The SPARC data is plotted along with theoretical NFW and ULDM profiles, assuming a virial mass of $5\times10^{11} \mathrm{M}_{\odot}$, $m_{22} = 0.1$, and $\pm 50 \%$ scatter in the ULDM core-halo mass relation and $\pm2\sigma$ scatter in NFW concentration. Galaxies in the legend are ordered from highest maximum velocity (top left) to lowest (bottom right).\re{red v blue?}}\label{fig:low_v}
\end{figure*}



\section{Conclusions}\label{sec:conclusion}

While the ULDM model has gained attention in part because it is a candidate solution to  the CDM core-cusp problem, in some cases ULDM profiles can have higher densities than their NFW counterparts at observationally relevant radii in the interior of halos of $M \gtrsim 10^{12} \mathrm{M}_{\odot}$. However,  the apparent spread in the predicted ULDM core-halo mass relation \cite{Schive:2014hza} leads to a sizeable range of plausible central densities for a halo of any given mass. Likewise, analyses of oscillations of the cores of ULDM halos on timescales much smaller than the relaxation time have also demonstrated significant fluctuations in central density \cite{Veltmaat:2018dfz}. This suggests that theoretical core-halo mass relations should not be interpreted too literally for any individual ULDM halo and limited simulation data means that the exact features of the correspondin statistical distribution are poorly characterised. Nevertheless, core masses at the lower end of the plausible range can mitigate the apparent worsening of the core-cusp discrepancy for ULDM halos.  

With the spread in the predicted core-halo relation in accounted for comparisons of theoretical ULDM and NFW profiles to  the SPARC database yield inconclusive results. While  parameters can be chosen to provide a superficial fit to given subsets of data, these data often do not span suitably large radial ranges for a meaningful assessment of the relative success of the theoretical UDLM and NFW profiles. This is because the regions in which the theoretical profiles differ strongly are often the same regions for which little observational data exists.  \re{this is unclear}

We conclude that the current quality of data and theoretical modelling is such that we cannot draw firm conclusions about the relative viability of ULDM and CDM. In principle, while one could perform a BIC analysis to distinguish between the two models, the lack of comprehensive data and the high number of free parameters (the stellar mass to light ratio in the SPARC data or assumed virial mass of the galactic halos,  ULDM particle mass, the NFW concentration parameter, the UDLM soliton to NFW transition radius and the variation around the core-halo mass relation) such an analysis would be highly vulnerable to overfitting.

It is thus clear that more information, both from simulations and astrophysical observations, is required to  assess the applicability of the ULDM model of dark matter. The parameter space describing ``typical'' ULDM halos appears to be larger than simple semi-analytical models would suggest, and must also include  baryonic feedback. Tightening the constraints on the plausible ULDM particle mass will also inform future investigations of this type \cite{Castellano:2019hdd, Lidz:2018fqo, Davoudiasl:2019nlo} Furthermore, a larger volume of photometric data with improved uncertainties covering a greater halo mass range and radius would be of tremendous benefit when testing dark matter scenarios, and can be expected from future surveys  \cite{Simon:2019kmm}. In addition, cosmological structure formation simulations for ULDM models are an active area of research \cite{Lin:2018whl, Clough:2018exo, Mocz:2015sda}, and these will improve the clarity of predictions for ULDM scenarios. %This work, along with previous studies such as \cite{Bar2018acw}, which also tackled the core-halo mass relation and the fitting of semi-analytical profiles to galaxy data, emphasise the necessarily preliminary and tentative nature of all analyses of ULDM-derived rotation curves, and provides clear targets for future analyses. 

\re{one final sentence}

\begin{acknowledgements}
We acknowledge invaluable discussions with Jens Niemeyer, Shaun Hotchkiss, and Mateja Gosenca in completing this work. We also acknowledge support from the Marsden Fund of the Royal Society of New Zealand. This research was supported by use of the Nectar Research Cloud, a collaborative Australian research platform supported by the National Collaborative Research Infrastructure Strategy (NCRIS).

\end{acknowledgements}

\re{

\begin{appendix}

\begin{figure*}[t]
\centering
\includegraphics[scale = 0.8, trim={0cm 2.5cm 1cm 0cm}]{000_comp_10_12.png} 
\caption{Plot demonstrating the effect of changing the ULDM particle mass assumption on the velocity profiles for halos of mass $10^{12}\mathrm{M}_{\odot}$.}\label{fig:vel_5_10_11}
\end{figure*}

\section{Core-halo mass relation}\label{app:core-halo}

 The core-halo mass relation can be simply interpreted as the statement that the average internal velocity of a tracer mass in the core must be equal to the virial velocity of a tracer mass in the wider halo. If this were not the case, and instead the average velocity were higher within the core, these higher velocity particles would move outward, resulting in dynamical mass redistribution within the halo. During this process, the halo would not be in equilibrium and would thus not be virialised.

From the virial theorem we have that $E_K=-1/2 \ E_P$, where $E_K$ and $E_P$ represent kinetic and potential energies, respectively. Alternatively we can write:

\begin{equation}
    \frac{1}{2}M_{tot}v^2=\frac{1}{4}\frac{GM_{tot}^2}{R_{tot}},
\end{equation}
where $G$ is the gravitational constant, $M_{tot}$ and $R_{tot}$ are the total mass and radius, and $v^2$ is the mean of the squares of individual tracer velocities. Demanding that $v^2$ is the same for the core as for the total virialised halo allows us to then write
\begin{align}\label{eq:virial_cond}
    v^2&=\frac{GM_{\mathrm{vir}}}{2 R_{\mathrm{vir}}}=\frac{G M_{core}}{2 R_{core}}\nonumber\\
    &\rightarrow R_{core}=\frac{M_{core} R_{\mathrm{vir}}}{M_{\mathrm{vir}}}.
\end{align}
We know from the soliton scaling properties that $R_{core}\propto M_{core}^{-1}$, and since $M_{\mathrm{vir}}=4/3 \ \pi R_{\mathrm{vir}}^3 \Bar{\rho}$, we also have $R_{\mathrm{vir}} \propto M_{\mathrm{vir}}^{1/3}$. Hence, Equation \ref{eq:virial_cond} becomes
\begin{align}
    &R_{core}^2\propto \frac{R_{\mathrm{vir}}}{M_{\mathrm{vir}}}\nonumber\\
    &\rightarrow R_{core}^2\propto \frac{M_{\mathrm{vir}}^{1/3}}{M_{\mathrm{vir}}}\nonumber\\
    &\rightarrow R_{core}\propto\left(M_{\mathrm{vir}}^{-2/3}\right)^{1/2}\nonumber\\
    &\rightarrow R_{core}\propto M_{\mathrm{vir}}^{-1/3}.
\end{align}
With this scaling relation in mind, the constant of proportionality may be determined through analysis of simulated halos. 



\section{ - Impact of ULDM particle mass on halo velocity profiles}

Figure \ref{fig:vel_5_10_11} demonstrates the scale of the changes to the velocity profiles of theoretical ULDM halos under changes in the ULDM particle mass. Further work to constrain the plausible range of the particle mass will make comparisons of the ULDM and CDM models with astrophysical data more effective.

 
\end{appendix}

\bibliographystyle{pasa-mnras}
\bibliography{1r_lamboo_notes}

\end{document}
